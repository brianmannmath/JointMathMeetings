 \documentclass{beamer}
 \usetheme{Madrid}


%%PACKAGES

\usepackage{amsmath}
\usepackage{amsfonts}
\usepackage{amssymb}
\usepackage{amsthm}
\usepackage{graphicx}
%\usepackage{diagrams}
\usepackage{mathrsfs}
\usepackage{manfnt}
%\usepackage{pstricks}

%%THEOREM STYLES

\theoremstyle{theorem}
\newtheorem{thm}{Theorem}
%\newtheorem{lemma}{Lemma}
\newtheorem{cor}{Corollary}
\newtheorem*{prelemma}{Lemma}
\newtheorem*{prethm}{Theorem}
\newtheorem{prop}{Proposition}
\newtheorem*{conj}{Conjecture}
\theoremstyle{definition}
\newtheorem*{defin}{Definition}
\newtheorem*{goal}{Goal}
\newtheorem*{remark}{Remark}
\newtheorem{ex}{Example}

\renewcommand{\baselinestretch}{1}

%%CUSTOM COMMANDS

\newcommand{\prob}[1]{\subsection*{#1}}
\newcommand{\subprob}[1]{\subsubsection*{#1}}


\newcommand{\R}{\ensuremath{\mathbb R}}
\newcommand{\A}{\ensuremath{\mathbb A}}
\newcommand{\Z}{\ensuremath{\mathbb Z}}
\newcommand{\C}{\ensuremath{\mathbb C}}
\newcommand{\Q}{\ensuremath{\mathbb Q}}
\newcommand{\T}{\ensuremath{\mathbb T}}
\renewcommand{\H}{\ensuremath{\mathbb H}}
\newcommand{\CP}{\ensuremath{\mathbf{CP}}}
\newcommand{\RP}{\ensuremath{\mathbf{RP}}}
\renewcommand{\P}{\ensuremath{\mathbf P}}
\newcommand{\eset}{\emptyset}

\newcommand{\isom}{\cong}
\renewcommand{\implies}{\Rightarrow}
\renewcommand{\iff}{\Leftrightarrow}

\renewcommand{\O}{\ensuremath{\mathcal{O}}}
\newcommand{\F}{\ensuremath{\mathcal F}}
\newcommand{\G}{\ensuremath{\mathcal{G}}}
\newcommand{\I}{\ensuremath{\mathscr{I}}}
\newcommand{\E}{\ensuremath{\mathcal{E}}}
\newcommand{\M}{\ensuremath{\mathcal{M}}}
\newcommand{\K}{\ensuremath{\mathcal{K}}}
\newcommand{\N}{\ensuremath{\mathbf{N}}}
\newcommand{\U}{\ensuremath{\mathfrak U}}

\newcommand{\cech}{\check{\text{C}}\text{ech}}

\renewcommand{\a}{\ensuremath{\mathfrak a}}
\newcommand{\m}{\ensuremath{\mathfrak m}}
\newcommand{\p}{\ensuremath{\mathfrak p}}
\newcommand{\q}{\ensuremath{\mathfrak q}}
\newcommand{\e}{\ensuremath{\epsilon}}

\newcommand{\vphi}{\varphi}


\newcommand{\Div}{\ensuremath{\operatorname{Div}}}
\newcommand{\spec}{\ensuremath{\operatorname{Spec}}}
\newcommand{\rank}{\ensuremath{\operatorname{rank}}}
\newcommand{\injrad}{\ensuremath{\operatorname{inj}}}
\newcommand{\vol}{\ensuremath{\operatorname{vol}}}
\newcommand{\End}{\ensuremath{\operatorname{End}}}
\newcommand{\iso}{\ensuremath{\operatorname{Isom}}}
\newcommand{\proj}{\ensuremath{\operatorname{Proj}}}
\newcommand{\hgt}{\operatorname{ht}}
\newcommand{\im}{\operatorname{Im}}
\renewcommand{\hom}{\operatorname{Hom}}
\newcommand{\adj}{\operatorname{Adj}}
\renewcommand{\-}{\ensuremath{^{-1}}}
\renewcommand{\>}{\ensuremath{\rightarrow}}
\newcommand{\map}{\ensuremath{\mapsto}}
\newcommand{\id}{\ensuremath{\mathbf{Id}}}
\newcommand{\pr}{^{\prime}}
\newcommand{\rad}{\operatorname{Rad}}
\newcommand{\ann}{\operatorname{Ann}}
\newcommand{\ass}{\operatorname{Ass}}
\newcommand{\supp}{\operatorname{Supp}}
\newcommand{\coker}{\operatorname{coker}}
\newcommand{\del}{\partial}


\renewcommand{\(}{\langle}
\renewcommand{\)}{\rangle}
\newcommand{\inject}{\hookrightarrow}
\newcommand{\surject}{\twoheadrightarrow}
\newcommand{\Aut}{\text{Aut}\,}
\newcommand{\Out}{\text{Out}}
%\newcommand{\End}{\text{End}\,}
\newcommand{\ad}{\text{ad}\,}
\newcommand{\Ad}{\text{Ad}\,}

\newcommand{\invlim}{\mathop{\varprojlim}\limits}
\newcommand{\directlim}{\mathop{\varinjlim}\limits}

\newcommand{\surj}{\twoheadrightarrow}
\newcommand{\inj}{\hookrightarrow}

\renewcommand{\part}[2]{\frac{\del{#1}}{\del{#2}}}

%MARGINS

%\usepackage[left=1in,top=1in,right=1in,bottom=1in]{geometry}

\AtBeginSection[] {
\begin{frame}<beamer>
   \frametitle{Sections}
   \tableofcontents[currentsection]
\end{frame}}

\title{Hyperbolic $Out(F_N)$-graphs}
\author{Brian Mann}
\institute{University of Utah}
\date{January 16, 2014}

\begin{document}

\begin{frame}
\titlepage
\end{frame}



\section{Preliminaries}



\begin{frame}
\frametitle{Hyperbolic metric spaces}

\begin{goal}
Study $Out(F_N)$ via its action on hyperbolic graphs.
\end{goal}

\pause

\begin{defin}
Let $\delta > 0$. A geodesic metric space $X$ is \emph{$\delta$-hyperbolic} if for any geodesic triangle with vertices $a,b,c$, the geodesic segment $ab$ is contained in a $\delta$-neighborhood of the union of $bc$ and $ac$.

If $X$ is $\delta$-hyperbolic for some $\delta$, we often just say $X$ is \emph{hyperbolic}. 
\end{defin}

\pause

Some well-known examples are: trees ($\R$-trees, if you know what those are) and hyperbolic space.
\end{frame}

\begin{frame}
\frametitle{Actions on hyperbolic spaces}
Let $X$ be a hyperbolic metric space.
\pause
\begin{defin}
An isometry $g$ of $X$ is called a \emph{hyperbolic isometry} or just \emph{hyperbolic} if it has an invariant bi-infinite quasi-geodesic $\gamma$ in $X$. We will call $\gamma$ a \emph{quasi-axis} for $g$.
\end{defin}
\pause
\end{frame}

\section{Weak Proper Discontinuity}

\begin{frame}
Let $G$ be a finitely presented group and $X$ a hyperbolic metric space.
\pause
\begin{defin}
An action of $G$ on $X$ satisfies \emph{Weak Proper Discontinuity (WPD)} if:
\pause
\begin{itemize}
\item $G$ is not virtually cyclic.
\pause
\item $G$ contains at least one hyperbolic isometry.
\pause
\item for each hyperbolic isometry $g \in G$, $x \in X$, and $C >0$ there exists an $N > 0$ such that $$\{  \in G | d(x,h(x)) \leq C, d(g^N(x), hg^N(x)) \leq C \}$$ is finite.
\end{itemize}
\end{defin}
\pause
In English: sufficiently long segments of quasi-axes for hyperbolic $g \in G$ are coarsely stabilized by only finitely many elements of $G$
\end{frame}

\begin{frame}
\begin{remark}
If the action of $G$ on $X$ satisfies WPD, then for $g \in G$ hyperbolic, the centralizer of $g$ is virtually cyclic.
\end{remark}
\pause
Some examples:
\pause
\begin{itemize}
\item The action of $MCG(S)$ on $\mathcal{C}(S)$. (Bestvina-Fujiwara \cite{BF02})
\pause
\item A hyperbolic group $G$ acting on its Cayley graph.
\end{itemize}
\pause
\begin{thm}[Sisto \cite{Sisto13}]
Hyperbolic isometries are generic in $G$ (in some sense involving a random walk on the Cayley graph).
\end{thm}
\end{frame}

\section{Hyperbolic $Out(F_N)$-graphs}

\begin{frame}
\frametitle{The free splitting graph}
\begin{defin}
The \emph{free splitting graph} $FS_N$ has vertex set $=$ one edge graph of groups decompositions of $F_N$ with trivial edge group (up to some equivalence). Two vertices are connected by an edge if there is a two-edge graph of groups decomposition collapsing to both.
\end{defin}
\pause
\begin{thm}[Handel-Mosher \cite{HandelMosher}]
$FS_N$ is hyperbolic.
\end{thm}
\end{frame}

\begin{frame}
The action of $Out(F_N)$ on $FS_N$ is not WPD! \\
\pause
Consider a surface $S = S_{g,1}$ with genus $g$ and $1$ puncture attached to a rose $R$ with $k+1$ petals by attaching the boundary of $S$ to one petal of $R$ by the identity.
\pause
\includegraphics[scale = .5]{WPD_1.jpg}
\end{frame}

\begin{frame}
\includegraphics[scale = .3]{WPD_1.jpg}

Choose an automorphism $\phi$ of $F_{2g + k}$ which maps $a_1, \ldots, a_k$ over $a_{k+1}$ in some complicated way and which is the identity on $S$. If we do this the right way, we ensure that $\phi$ acts hyperbolically on $FS_N$. But any surface automorphism of $S$ commutes with $\phi$ since it is the identity on $\del S$, so the centralizer of $\phi$ is not virtually cyclic!
\end{frame}

\begin{frame}
\frametitle{The Cyclic Splitting Graph}

\begin{defin}
The \emph{cyclic splitting graph} $FZ_N$ has vertex set $=$ one-edge graph of group decompositions of $F_N$ with cyclic ($\Z$ or trivial) edge groups. Two vertices are connected by an edge if there exists a two-edge decomposition collapsing to both.
\end{defin}
\pause
\begin{thm}[- 2013 \cite{Mann13}]
$FZ_N$ is hyperbolic.
\end{thm}
\pause
Q: Is it WPD?
\pause
Note that the automorphisms above where WPD fails for $FS_N$ fix a cyclic splitting.
\end{frame}

\begin{frame}
\frametitle{The free factor graph}
\begin{defin}
The \emph{free factor graph} $FF_N$ has vertex set $=$ one-edge graph of groups decompositions of $F_N$ with trivial edge group. Two vertices are connected by an edge if there exists a proper free factor which (up to conjugacy) contained in vertex groups of both decompositions.
\end{defin}
\pause
\begin{thm}[Bestvina-Feighn \cite{BF09}, \cite{BF11}]
$FF_N$ is hyperbolic and the action of $Out(F_N)$ is WPD.
\end{thm}
\end{frame}

\begin{frame}
\frametitle{The Intersection Graph}
\begin{defin}
The \emph{intersection graph} $I_N$ has vertex set $=$ one-edge graph of groups decompositions of $F_N$ with cyclic edge group. Two vertices are connected by an edge if the corresponding Bass-Serre trees share a common elliptic conjugacy class (which might not be contained in a free factor! $[a,b] \in F_2$). 
\end{defin}
\pause
\begin{thm}[- 2013]
$I_N$ is hyperbolic and the action of $Out(F_N)$ is WPD. The hyperbolic isometries are fully irreducible automorphisms which have no periodic conjugacy classes.
\end{thm}
\pause
\begin{cor}
Fully irreducible automorphisms with no periodic conjugacy classes are generic in $Out(F_N)$.
\end{cor}
\end{frame}

\bibliographystyle{plain}
\bibliography{Refs,indecompREF}
\end{document}
















\end{document}

